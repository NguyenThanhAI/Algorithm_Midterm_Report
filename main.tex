\documentclass[14pt, a4paper]{article}
\usepackage{minitoc}
\usepackage[left=3.00cm, right=2.5cm, top=2.00cm, bottom=2.00cm]{geometry}
\usepackage{amsmath}
\usepackage{amssymb}
\usepackage{amsthm}
\usepackage{thmtools}
\usepackage{mathtools}
\usepackage{graphicx}
%\usepackage{algpseudocode}
%\usepackage{algorithm}
\usepackage[ruled,vlined,linesnumbered,algosection]{algorithm2e}
\usepackage{blindtext}
\usepackage{setspace}
\usepackage[utf8]{inputenc}
\usepackage[utf8]{vietnam}
\usepackage[center]{caption}
\usepackage[shortlabels]{enumitem}
\usepackage{fancyhdr} % header, footer
\usepackage{hyperref} % loại bỏ border với mục lục và công thức
\usepackage[nonumberlist, nopostdot, nogroupskip]{glossaries}
\usepackage{glossary-superragged}
\usepackage{tikz,tkz-tab}
\setglossarystyle{superraggedheaderborder}
\pagestyle{fancy}
%\usepackage[style=numeric,sortcites]{biblatex}
%\addbibresource{ref.bib}
%\usepackage[numbers]{natbib}
\usepackage{indentfirst}
\usepackage{multirow}
\usepackage[natbib,backend=biber,style=ieee, sorting=ynt]{biblatex}
\bibliography{ref.bib}

\graphicspath{{./figures/}}


\makenoidxglossaries

% Danh mục thuật ngữ


\hypersetup{
    colorlinks=false,
    pdfborder={0 0 0},
}


\fancyhf{}
\rhead{\textbf{Môn học: Toán rời rạc và thuật toán}}
\lhead{\textbf{GVHD: PGS. TS. Nguyễn Thị Hồng Minh}}
\rfoot{\thepage}
\lfoot{\textbf{Học viên thực hiện: Nguyễn Chí Thanh - 21007925}}
\renewcommand{\headrulewidth}{0.4pt}
\renewcommand{\footrulewidth}{0.4pt}


\numberwithin{equation}{section}
\numberwithin{figure}{section}

\setlength{\parindent}{0.5cm}

\setcounter{secnumdepth}{3} % Cho phép subsubsection trong report
\setcounter{tocdepth}{3} % Chèn subsubsection vào bảng mục lục

\newtheorem{dl}{Định lý}
\newtheorem{md}{Mệnh đề}
\newtheorem{bd}{Bổ đề}
\newtheorem{dn}{Định nghĩa}
\newtheorem{hq}{Hệ quả}

\numberwithin{dl}{section}
\numberwithin{md}{section}
\numberwithin{bd}{section}
\numberwithin{dn}{section}
\numberwithin{hq}{section}

\doublespacing
\AtBeginEnvironment{tabular}{\doublespacing}

\begin{document}
    \begin{titlepage}

        \newcommand{\HRule}{\rule{\linewidth}{0.5mm}} % Defines a new command for the horizontal lines, change thickness here

        \center % Center everything on the page

        %----------------------------------------------------------------------------------------
        %	HEADING SECTIONS
        %----------------------------------------------------------------------------------------
        \textsc{\LARGE Đại học Quốc Gia Hà Nội}\\[0.5cm]
        \textsc{\LARGE Trường đại học Khoa học tự nhiên}\\[0.5cm] % Name of your university/college
        \textsc{\LARGE Khoa Toán - Cơ - Tin học}\\[0.5cm]

        \includegraphics[scale=0.2]{HUS-logo.jpg}\\[0.5cm]

        \textsc{\Large Chuyên ngành: Khoa học dữ liệu}\\[0.5cm] % Major heading such as course name


        %----------------------------------------------------------------------------------------
        %	TITLE SECTION
        %----------------------------------------------------------------------------------------

        \HRule \\[0.4cm]
        { \huge \bfseries THU HOẠCH THỰC TẾ MÔN HỌC}\\[0.4cm] % Title of your document
        \HRule \\[1.5cm]

        \textsc{\Large Môn học: Toán rời rạc và thuật toán}\\[1cm] % Minor heading such as course title


        %\textsc{\Large Đề tài: Ước lượng ma trận hiệp phương sai tối ưu với \\ trung bình không hoàn hảo \\ trong mô hình khuếch toán xác suất}\\[2cm]


        %----------------------------------------------------------------------------------------
        %	AUTHOR SECTION
        %----------------------------------------------------------------------------------------
        \begin{minipage}{0.4\textwidth}
            \begin{flushleft} \large
            \emph{Giảng viên hướng dẫn:} \\
            PGS. TS. Nguyễn Thị Hồng Minh % Supervisor's Name
            \end{flushleft}
        \end{minipage}\\[0.5cm]

        \begin{minipage}{0.4\textwidth}
        \begin{flushleft} \large
        \emph{Học viên thực hiện:}\\
        Nguyễn Chí Thanh \\
        MSHV: 21007925 \\ % Your name
        Lớp: Khoa học dữ liệu - K4
        \end{flushleft}
        \end{minipage}


        % If you don't want a supervisor, uncomment the two lines below and remove the section above
        %\Large \emph{Author:}\\
        %John \textsc{Smith}\\[3cm] % Your name

        %----------------------------------------------------------------------------------------
        %	DATE SECTION
        %----------------------------------------------------------------------------------------

        % I don't want day because it is English
        % {\large \today}\\[2cm] % Date, change the \today to a set date if you want to be precise

        %----------------------------------------------------------------------------------------
        %	LOGO SECTION
        %----------------------------------------------------------------------------------------

        %\includegraphics{logo/rsz_3logo-khtn.png}\\[1cm] % Include a department/university logo - this will require the graphicx package

        %----------------------------------------------------------------------------------------

        \vfill % Fill the rest of the page with whitespace

    \end{titlepage}

    \cleardoublepage
    \pagenumbering{gobble}
    \tableofcontents
    \newpage
    \listoffigures
    \newpage
    \glsaddall 
    \renewcommand*{\glossaryname}{Danh mục các từ viết tắt}
    \renewcommand*{\acronymname}{Danh sách từ viết tắt}
    \renewcommand*{\entryname}{Viết tắt}
    \renewcommand*{\descriptionname}{Viết đầy đủ}
    \printnoidxglossary
    \cleardoublepage
    \pagenumbering{arabic}

    %\maketitle

    \newpage

    \nocite{*}

    \begin{center}
    \section*{LỜI MỞ ĐẦU}
    \end{center}
    \addcontentsline{toc}{section}{{\bf LỜI MỞ ĐẦU}\rm}

    \newpage

    \section{Các bài thuyết trình hướng nghiệp}

    \subsection{Các cơ hội nghề nghiệp và định hướng trong thị trường CNTT}

    Bài thuyết trình đầu tiên được thực hiện bởi ông Nguyễn Trung Kiên, Trưởng phòng phát triển phần mềm, Công ty Optimizely Việt Nam.
    Công ty Optimizely được thành lập năm 1994 tại Thụy Điển. Trụ sở chính tại New York - Mỹ, có văn phòng ở nhiều nơi trên thế giới, chủ yếu ở Mỹ và châu Âu.
    Optimizely đã được Garner vinh danh top 3 giải pháp tốt nhất về mảng DXP, CMS, Marketing automation liên tục từ năm 2019 đến năm 2022.
    Optimizely Vietnam đã hoạt động được 25 năm, là môi trường làm việc quốc tế chuyên nghiệp và sáng tạo, được ITViec vinh danh top 15 công ty tốt nhất thị trường Việt Nam.

    \subsubsection{Bức tranh tổng quát CNTT 2021-2025}

    Giá trị nền kinh tế Internet từ 21 tỷ USD (2021) tăng trưởng lên 57 tỷ USD (2030).
    Tốc độ tăng trưởng thị trường 17\% (2020), 31 \% (2021), dự kiến tốc độ bình quân 29 \% đến 2025.

    Các ngành nghề khởi nghiệp được ưu tiên phát triển/hỗ trợ đến 2030 bao gồm:

    
    \begin{itemize}
        \item Các công ty phát triển công nghệ lõi.
        \item Các công ty phát triển các sản phẩm và dịch vụ công nghệ kỹ thuật số.
        \item Các công ty phát triển các giải pháp công nghệ kỹ thuật số.
        \item Khởi nghiệp công nghệ số
    \end{itemize}

    Một thống kê nhân sự trong ngành IT: tuổi nghề IT còn khá trẻ, số nhân sự dưới 30 tuổi chiếm tới 54 \%.
    Số năm kinh nghiệm 3 năm trở lên chiếm 30 \% dưới 3 năm là 52 \%.
    Phần lớn nhân sự trong ngành là nam giới với 90 \%.
    Phân bố chủ yếu ở 2 thành phố lớn là Hà Nội và thành phố Hồ Chí Minh vứi tỷ lệ 89,4 \%.

    Các công nghệ hay được sử dụng:

    \begin{itemize}
        \item Top 5 ngôn ngữ lập trình: Javascript, Java, NodeJS, PHP, C\#
        \item Top 5 database: MySQL, SQL Server, MongoDB, PostgreSQL, Redis
        \item Top 4 DevOps: Linux, DOcker, Bash, Kubernetes
        \item Top 5 Cloud platform: AWS, Azure (Microsoft), VMWare, Firebase, Google cloud
    \end{itemize}

    Một số vị trí khó tìm ứng viên như Fullstack, Devops, Backend

    Một số hình thức đánh giá năng lực kỹ thuật:

    \begin{itemize}
        \item Kiểm tra đánh giá với vấn đề - dự án thực tế.
        \item Phỏng vấn code trực tiếp - tại chỗ.
        \item Pair programming
        \item Cuộc thi về tech thông qua các thử thách - trò chơi
        \item Whiteboard coding test.
        \item Kiểm tra đánh giá tại nhà
    \end{itemize}

    Một số lưu ý khi giao tiếp với nhà tuyển dụng:

    \begin{itemize}
        \item Dưới 10s là thời gian nhà tuyển dụng dành để xem CV.
        \item Ứng viên cần chú ý những lỗi căn bản trong CV: Lỗi đánh máy, email không chuyên nghiệp, ảnh không phù hợp, hồ sơ facebook,
        \item Bên cạnh chuyên môn, kỹ năng mềm, nhà tuyển dụng sẽ lưu ý các điểm sau:
        \begin{itemize}
            \item Phù hợp văn hóa
            \item Phù hợp với vai trò và đội ngũ
            \item Phong cách giao tiếp
        \end{itemize}
    \end{itemize}

    \subsubsection{Data Scientist - Data Engineering}

    Data Scientist - Data Engineering là người làm việc với dữ liệu, sử dụng các kỹ năng kỹ thuật, kỹ năng phân tích để xác định các mẫu, xử lý dữ liệu và rút ra kết luận có giá trị.

    Data Scientist tập trung vào lấy mẫu, dựng mô hình dữ liệu, dự đoán kết quả

    Data Engineering tập rung xây dưng ứng dụng data pipelines thực hiện mô hình và giao diện người dùng cuối

    Kỹ năng: Machine Learning, tạo mô hình dữ liệu, Python, hiểu biết về kinh doanh, hoạt động doanh nghiệp, big data, hadoop, map reduce,\dots

    \subsubsection{Fullstack - Blockchain Developer}

    Fullstack Developer là lập trình viên web toàn diện, sử dụng thông thạo nhiều ngon ngữ, công cụ của cả frontend và backend cũng như cơ sở dữ liệu.

    Blockchain Developer là người chịu trách nhiệm phát triển và cải tiến các ứng dụng liên quan đến blockchain, nổi tiếng là dApps (Decentralized Applications), hợp đồng thông minh (smart contract), thiết kế kiến trúc và giao thức blockchain.

    Kỹ năng: ngôn ngữ lập trình (C++, Java, Python, Javascript,...) thuật toán, cấu trúc dữ liệu, cấu trúc phần mềm, smart contract, an ninh bảo mật, mã hóa,...

    \subsection{Blockchain is eating you}

    Bài thuyết trình được thực hiện bởi Giáo sư David Trần, Đại học Massachusetts, Mỹ. 
    Giáo sư David Trần đã trình bày về xu hướng phổ biến tất yếu của Blockchain cũng như các nền tảng toán học cơ bản của Blockchain.

    Một số bài toán tiêu biểu trong Blockchain:

    \begin{enumerate}
        \item Trao đổi tài sản không qua bên thứ 3
        Alice muốn chuyển cho Bob 100 USD để đổi lấy 2 triệu VNĐ.
        Một cách tự nhiên, Alice gửi 100 \$ đến Bob và hy vọng Bob sẽ trả lại 2 triệu VNĐ.
        Nhưng trong thế giới thực, Bob có thể lấy 100 \$ rồi biến mất.

        Giải pháp Atomic Swap: để đảm bảo giuao dịch thành công hoặc không xảy ra mất mát tài sản mà không cần bên thứ 3.

        Chứng minh định lý Zero-Knowledge: Một cách tổng quát, Alice có thể cung cấp một chứng minh rằng cô ấy biết giá trị $x$ ví dụ cho trước $f(x)=c$ và $c$ biết trước mà không cần tiết lộ giá trị $x$.
        Định lý này là định lý Zero-Knowledge Proof (ZK-Proof).
        Định lý này vô cùng quan trọng đối với tương lai của blockchain.
        Nếu triển khai thành công, sẽ làm cho tốc độ Blockchain nhanh hơn 1000 lần so với hiện tại.

        \item Bài toán cơ chế đồng thuận phi tập trung trong Blockchain
        Những đứa trẻ có bùn trên trán. Có 10 đứa trẻ chơi ngoài sân, 2 trong số 10 đứa trẻ có bùn trên trán (chỉ có cô giáo mới thấy). Từng đứa trẻ có thể nhìn thấy trán của những đứa khác nhưng không thể thấy trán của chính mình.
        Cô giáo yêu cầu: "Ai có bùn trên trán tự giác bước lên trên". Bài toán là làm thế nào từng đứa trẻ biết được có bùn trên trán mà không cần giao tiếp với những đứa trẻ khác?

        Lời giải: 
        \begin{itemize}
            \item Nếu có $k$ đứa trẻ có bùn trên trán sẽ cần $k$ lần mỗi lần một đứa trẻ bước lên trên.
            \item Nếu $k=2$
        \end{itemize}

        \begin{itemize}
            \item Nếu $k=1$ (Alice bị bùn trên trán): Alice sẽ bước lên vì Alice không có đứa trẻ nào khác có bùn.
            \item Nếu $k=2$ (Alice và Bob bị bùn trên trán): Vòng 1: Alice không bước lên trên vì thấy có Bob cũng có bùn trên trán nhưng cô ấy không chắc mình cũng bị bùn trên trán (Bob cũng nghĩ vậy).
            Vòng 2: Bob không bước lên ở vòng 1, nhưng Alice biết Bob nhìn thấy một người khác có bùn trên trán và cô ấy có thể là đứa trẻ đó. Nhưng cô ấy không thấy ai nữa có bùn trên trán ngoài Bob, người đó phải là cô ấy. Và vì vậy Alice bước lên (Bob cũng nghĩ vậy và bước lên)
            \item Tổng quát, với $k < n$ chỉ sau $k$ vòng tất cả những đứa trẻ có bùn trên trán sẽ bước lên.
        \end{itemize}

        Lịch sử của bài toán cơ chế đồng thuận:

        \begin{itemize}
            \item Năm 1970: Bài toán điều kiển máy bay. Máy tính được sử dụng trong điều khiển máy bay.
            Trong một hệ thống cần sự an toàn cao, hệ thống cần được vận hành trên nhiều máy tính để tăng độ tin cậy.
            NASA đã tài trợ hệ thống kiểm soát lỗi dựa trên phần mềm để xây dựng hệ thống điều kiển máy bay.
            Trong dự án này, Lamport đã đề xuất bài toán "Các vị tướng Byzantine" và đặt nền móng cho cơ chế đồng thuận phi tập trung.
            \item Năm 2000: Ứng dụng vào công nghiệp. Các công ty như Google và Facebook đã ứng dụng cơ chế đồng thuận phi tập trung trong các dịch vụ mang tính quan trọng như là Google Wallet và Facebook Credit.
            \item Năm 2009: Bitcoin là một đột phá trong cơ chế đồng thuận phi tập trung, bitcoin đã thể hiện rằng cơ chế đồng thuận hoàn toàn là khả thi trong phi tập trung, mà mọi người đều có thể tham gia trong môi trường không đáng tin cậy
        \end{itemize}

        Bài toán các vị tướng Byzantine được đưa ra bởi 3 nhà khoa học máy tính Leslie Lamport, Robert Shostak và Marshall Pease trong một báo cáo khoa học mang tên "The Byzantine Generals Problem" vào năm 1982. Đây là bài toán tổng quát hoá của bài toán 2 vị tướng quân.

        Bài toán các vị tướng Byzantine miêu tả về một nhóm các vị tướng trong đội quân Byzantine (quân đội đế quốc Đông La Mã), tiến hành vây hãm 1 thành phố. Các vị tướng cần trao đổi để đạt được đến 1 thoả thuận về kế hoạch tấn công thành phố đó. Trong trường hợp đơn giản nhất, họ thoả thuận về việc nên tấn công hay rút lui. Một số có thể muốn tấn công, nhưng một số thì lại muốn rút lui, và vấn đề là nếu chỉ có một bộ phận tấn công riêng lẻ, thì họ sẽ gặp thất bại, và đó là kế hoạch tồi tệ hơn việc cùng tấn công hoặc cùng rút lui.

        Mọi thứ sẽ trở nên phức tạp khi mà một vị tướng sẽ có thể gửi tin nhắn đi cho các vị tướng khác. Chẳng hạn như trong trường hợp có 5 vị tướng, 2 ông đã phát tín hiệu muốn tấn công, 2 ông đã phát tín hiệu muốn rút lui, còn ông thứ 5 lại chơi trò 2 mang, nhắn với 2 ông muốn tấn công rằng mình muốn tấn công, còn nhắn với 2 ông muốn rút lui rằng mình sẽ rút lui. Khi đó phe tấn công nghĩ rằng tấn công là lựa chọn đa số và họ tấn công (và sẽ thất bại), phe rút lui thì nghĩ rằng rút lui là lựa chọn đa số và họ rút lui. Họ đã không đạt được sự đồng thuận (về việc có cùng ý kiến).

        Mọi thứ sẽ còn phức tạp hơn nữa khi ta đặt trong trường hợp họ còn phải gửi tin nhắn thông qua một người đưa thư, nên hoàn toàn có khả năng xảy ra tình trạng người đưa thư ... bị bắt, thư không gửi được đến nơi, hay nội dung message bị sửa đổi.

        Có khá nhiều giải pháp đã được đề cập trong báo cáo khoa học của Lamport, Shostak và Pease. Họ bắt đầu bằng một chú ý rằng, bài toán các vị tướng Byzantine, có thể giải quyết bằng cách quy về bài toán "Commander and Lieutenants" (chỉ huy và trung uý).

        Bài toán như sau:

        \begin{itemize}
            \item Người chỉ huy sẽ gửi lệnh cho các trung uý còn lại
            \item Những vị trung uý có thể gửi các message cho nhau
            \item Làm thế nào để những người trung thành có thể đạt được sự đồng thuận về một quyết định nào đấy (đơn giản như là tấn công hay rút lui)
        \end{itemize}

        Chú ý là kể cả trong trường hợp người chỉ huy là kẻ phản bội, thì tất cả (những vị tướng trung thành) vẫn vẫn cần đạt đến 1 sự đồng thuận.

        3 nhà khoa học máy tính Lamport, Shostak và Pease có đưa ra thuật toán Oral Messages (OM) để giải quyết vấn đề này. Để cho tất cả đạt được 1 sự đồng thuận, ta cần dựa vào sự lựa chọn của số đông.

        Tuy nhiên, trước hết cần giả định rằng:

        \begin{itemize}
            \item Mọi message khi được gửi, đều sẽ được gửi đến đích một cách chính xác
            \item Người nhận message sẽ biết được chính xác họ nhận từ ai
            \item Có thể phát hiện ra sự vắng mặt của một message (chẳng hạn như ai đó không gửi)
        \end{itemize}

        \begin{dl}
            Với mỗi giá trị $m$ là số lượng kẻ phản bội, thuật toán $OM(m)$ có thể đạt được sự đồng thuận nếu có tổng cộng là hơn $3m$ số lượng các vị tướng quân.
        \end{dl}

        Hay nói cách khác, nếu có tất cả là n vị chỉ huy, thì thuật toán OM sẽ đạt được sự đồng thuận khi có 2/3 là trung thành (hay không quá 1/3 là phản bội)

        Để dễ tưởng tượng, chúng ta hãy cùng xem qua trường hợp đơn giản, với 4 vị tướng (gồm C, L1, L2, L3), và 1 kẻ phản bội, như sau:

        \begin{figure}[h!]
            \centering
            \includegraphics{commander_leutenants.png}
            \caption{Các trường hợp xảy ra trong bài toán chỉ huy và trung úy}
        \end{figure}
        \begin{itemize}
            \item Trường hợp 1, kẻ phản bội là L3. C sẽ gửi tin với nội dung v cho L1, L2, L3. L3 phản bội nên sẽ sửa đổi nội dung, và gửi x cho L2. Tuy nhiên, L2 sẽ nhận được v từ L1 và C và sẽ thấy là phần đông đã lựa chọn v. Từ đó những vị tướng trung thành, gồm C, L1 và L2 sẽ đạt được sự đồng thuận là phương án v, mặc cho L3 có gửi tin x.
            \item Trường hợp 2, kẻ phản bội là C. C có thể gửi x cho L1, gửi y cho L2 và gửi z cho L3. L1, L2, L3 đều trung thành, nên sẽ gửi tin mà họ nhận được cho những người khác. Như vậy L1 sẽ nhận được đủ cả 3 lệnh là x (từ C), y (từ L2) và z (từ L3). Trông có vẻ như không thể quyết định được đấy, nhưng thực chất, quyết định của cả 3 người sẽ là giống nhau, vì cùng là majority(x,y,z). Nếu x, y, z mang những nội dung hoàn toàn khác nhau, và không thể tính trọng số, tất cả sẽ theo lựa chọn default, ở đây có thể là rút quân chẳng hạn.
        \end{itemize}
    \end{enumerate}

    \subsection{Sinh viên Khoa Toán - Cơ - Tin học và con đường phát triển nghề nghiệp tại một tổ chức tài chính ngân hàng}
    
    Bài thuyết trình được thực hiện bởi ông Lê Công Bình, Trưởng Phòng Triển khai và Phát triển ứng dụng, Trung tâm CNTT ngân hàng BIDV.
    Ngân hàng BIDV là ngân hàng có vốn hóa lớn nhất Việt Nam.

    Các vị trí công việc trong phòng CNTT tại ngân hàng BIDV:

    \begin{itemize}
        \item Kế hoạch chiến lược:
        \begin{itemize}
            \item Kế hoạch chiến lược
            \item Quản lý dự án CNTT
            \item Quản lý kiến trúc tổng thể
            \item Nghiên cứu công nghệ mới
        \end{itemize}
        \item Phát triển ứng dụng
        \begin{itemize}
            \item Phân tích nghiệp vụ (BA)
            \item Lập trình viên (Frontend, Mid, Backend)
            \item Phân tích thiết Kế
            \item Tích hợp hệ thống
            \item Đảm bảo chất lượng QA - QC
        \end{itemize}
        \item Phân tích dữ liệu
        \begin{itemize}
            \item Báo cáo quản trị điều hành
            \item Data Governance
            \item Phân tích dữ liệu khách hàng
            \item Phân tích dữ liệu rủi ro
        \end{itemize}
        \item Quản trị hạ tầng
        \begin{itemize}
            \item Quản trị máy chủ, thiết bị lưu trữ
            \item Quản trị cơ sở dữ liệu
            \item Quản trị mạng, truyền thông
            \item Quản trị an ninh bảo mật
            \item Quản trị trung tâm dữ liệu
        \end{itemize}
    \end{itemize}

    Một số vị trí cho lập trình viên và các công nghệ hay được sử dụng tại các vị trí tương ứng:

    \begin{itemize}
        \item Chuyên viên phát triển phần mềm: ReactJS, AngularJS, SpringBoot, Sonarqube
        \item Chuyên viên tích hợp hệ thống
        \item Chuyên viên thiết kế hệ thống: IBM WebSphere, IBM Rational Software
        \item Chuyên viên cấp cao
        \item Chuyên gia
    \end{itemize}

    Các kỹ năng và kiến thức mà sinh viên cần chuẩn bị:

    \begin{itemize}
        \item Tinh thần học hỏi và liên tục phát triển bản thân
        \item Tự lập trình từ đầu đến cuối theo các chuẩn framework
        \item Kiến thức cơ bản cơ sở dữ liệu và thiết kế cơ sở dữ liệu
        \item Tìm hiểu một số xu thế công nghệ: Microservice, kiến trúc mở, cloud, Big Data
        \item Kỹ năng đọc hiểu tiếng Anh - Giao tiếp cơ bản. Mục tiêu TOEIC 600 điểm hoặc tương đương
        \item Kỹ năng mềm: Phối hợp làm việc nhóm, quản lý thời gian, xác lập mục tiêu cá nhân trong phát triển nghề nghiệp
    \end{itemize}
    \section{Tham quan hội trại}

    \section{Chuỗi bài giản về khoa học dữ liệu}

    \subsection{Dự án giải mã 1000 hệ gen người Việt}

    Bài thuyết trình được thực hiện bởi TS. Võ Sỹ Nam, công ty GeneStory trình bày về dự án "Giải mã 1000 hệ gen người Việt".
    Bài thuyết trình đã đề cập đến lý thuyết cơ bản về DNA bao gồm: lịch sử ra đời mô hình, quá trình giải mã, các công trình nghiên cứu về nguồn gốc loài người dựa trên các công trình nghiên cứu về gen cũng như vai trò của việc giải mã gen trong kỷ nguyên của y học chính xác.

    Cấu trúc DNA được James Watson và Francis Crick đề xuất trong quá trình nghiên cứu tại Lab Cavendish năm 1953.
    Có nhiều thông tin cho rằng James Watson và Francis Crick đã lấy ý tưởng từ nhà nghiên cứu Rosalind Franklin tại Paris.
    DNA có cấu trúc xoắn kép. DNA chứa đựng các thông tin di truyền từ thế hệ này sang thế hệ khác nhờ khả năng phân đôi trong quá trình sinh sản và quyết định tất cả các đặc điểm của chúng ta. DNA có ở trong nhân tế vào và một lượng nhỏ nằm trong ty thể. 
    
    Thông tin trong DNA được lưu trữ dưới dạng mã, được tạo thành từ bốn loại bazơ nitơ là: adenine (A), guanine (G), cytosine (C) và thymine (T). Các bazơ này bắt cặp với nhau, A với T và C với G, thông qua các liên kết hydro để tạo thành các đơn vị được gọi là cặp bazơ.

    DNA có cấu trúc không gian dạng xoắn kép với 2 mạch song song. Thực tế, 2 mạch này xoắn đều xung quanh 1 mạch cố định và theo chiều ngược kim đồng hồ. Cấu trúc xoắn kép DNA của mỗi người là khác nhau, do đó mỗi chúng ta đều có các đặc điểm riêng biệt. Do có tính đặc thù nên nhờ phân tích DNA các nhà khoa học có thể khám phá ra sự phát triển và tiến hóa của mỗi giống loài cũng như tìm ra giải pháp tối ưu để hạn chế, điều trị các căn bệnh do đột biến DNA di truyền.

    Ngoài ra việc nghiên cứu DNA còn có thể giúp chúng ta tìm ra nguồn gốc loài người cũng như quá trình di cư của tổ tiên chúng ta.
    Công trình của các nhà khoa học tại Trung tâm nghiên cứu dữ liệu lớn tại Vingroup cho thấy di truyển người Việt có độ đa dạng cao

    \begin{figure}[h!]
        \centering
        \includegraphics{immgrants_process.png}
        \caption{Quá trình di cư của loài người qua công trình nghiên cứu DNA}
    \end{figure}

    \begin{figure}[h!]
        \centering
        \includegraphics{Vietnam_Extensive_Genetic_Diversity.png}
        \caption{Độ đa dạng di truyền của người Việt}
    \end{figure}

    Với các tiến bộ về khoa học kỹ thuật, giải mã gen một người chỉ tốn khoảng 200 \$ và sẽ sớm đạt được 100 \$ (cách đây khoảng 20 năm chi phí này là hàng chục triệu \$)
    Ngoài ra việc giải mã gen người còn đặt ra những thách thức về công nghệ lưu trữ dữ liệu: Để giải mã gen 1008 người Việt cần dung lượng khoảng 1200 TB dữ liệu với 37 \% đến từ miền Bắc, 22 \% đến từ miền Trung, 41 \% đến từ miền Nam.
    
    \begin{figure}[h!]
        \centering
        \includegraphics{Wetlab_to_Drylab.png}
        \caption{Quy trình giải mã gen từ máy giải trình tự đến các phân tích về gen}
    \end{figure}

    Để có thể làm việc với một lượng dữ liệu khổng lồ từ quá trình giải trình tự ta cần một nền tảng chuyên dụng cho dữ liệu lớn gồm các thành phần như: Apache Spark, Hadoop, Cassandra, Mesos, Apache HBase, Kubernetes, Elasticsearch.
    Các dữ liệu của công trình được chia làm hai phiên bản: Phiên bản 1 vào tháng 12/2020 với 58640 GB dữ liệu công khai và khoảng 450000 GB dữ liệu riêng tư. Phiên bản 2 vào tháng 12/2021 với 109308 GB dữ liệu mở và khoảng 1100000 GB dữ liệu riêng tư.
    \subsection{Ứng dụng KHDL hỗ trợ khách hàng ra quyết định đầu tư thông minh tại TCBS (Techcom Securities)}

    \subsection{Data Analytics: Leading continuous improvement and predicting trends}

    \subsection{Ứng dụng của toán học hiện đại và khoa học dữ liệu để giải quyết các bài toán trong lĩnh vực chuỗi cung ứng}
\end{document}